\documentclass[UTF8]{ctexart}
\usepackage{CJKutf8}
\usepackage{listings}
\usepackage{geometry}
\usepackage{xcolor}
\usepackage{graphics}
\usepackage{graphicx}
\lstset{
	backgroundcolor=\color{white},
	basicstyle=\footnotesize\ttfamily,
	breakatwhitespace=false,
	breaklines=true,
	captionpos=b,
	commentstyle=\color{mygreen},
	deletekeywords={...},
	escapeinside={\%*}{*)},
	extendedchars=true,
	frame=single,
	keepspaces=true,
	keywordstyle=\color{blue},
	morekeywords={*,...},
	numbers=left,
	numbersep=5pt,
	numberstyle=\tiny\color{gray},
	rulecolor=\color{black},
	showspaces=false,
	showstringspaces=false,
	showtabs=false,
	stepnumber=4,
	stringstyle=\color{purple},
	tabsize=4,
	title=\lstname
}
\geometry{a4paper, left=1cm, right=1cm, bottom=1cm}
\title{操作系统实验报告}
\author{顾芃骐20079100001}
\begin{document}
	\maketitle
	\newpage
	\section{进程的建立}
	\begin{itemize}
		\item 实验目的:学会通过基本的Windows或者Linux进程控制函数,由父进程创建子进程,并实现父子进程协同工作。
		\item 实验软件环境:Ubuntu22.04 \& GNU-C++17
		\item 实验内容:创建两个进程,让子进程读取一个文件,父进程等待子进程读取完文件后继续执行,实现进程协同工作。进程协同工作就是协调好两个进程,使之安排好先后次序并以此执行,可以用等待 函数来实现这一点。当需要等待子进程运行结束时,可在父进程中调用等待函数。
	\end{itemize}
	\subsection{代码实现}
    在“\textbf{\textit{os1.cpp}}”中写如下代码,并创建名为“\textbf{\textit{os1.in}}”的文件,在后者中写入且仅写入“TextMessage”作为测试输出内容。
\begin{lstlisting}[language=c++]
#include<fstream>
#include<unistd.h>
#include<stdio.h>
#include<iostream>
using namespace std;
int main(){
	int flag = 0;
	string answer = "";
	auto x = vfork();
	if (!x) {
		ifstream in("os1.in");
		in >> answer;
		in.close();
		flag = 1;
		cout << "End of Child Process" << endl;
	} else {
		while (!flag);
		cout << "Start of Main Process" << endl;
		cout << answer << endl;
		cout << "End of Main Process" << endl;
	}
	return EXIT_SUCCESS;
}
\end{lstlisting}
\subsection{实验结果}
运行上述代码,可得到如下左图所示的实验结果。片刻之后,终端显示如右所示的结果。
\begin{center}
        \includegraphics[width=0.3\pdfpagewidth]{os1-1.png}
        \includegraphics[width=0.3\pdfpagewidth]{os1-2.png}
\end{center}
\subsection{实验结果分析}
在1.1所示代码中,子进程读取了文件“\textit{os1.in}”;读取完成后,父进程继续执行,并显示了开始、结束和子进程所读取的文件内容。需要注意的是,使用\texttt{vfork}时修改静态区变量是未定义行为,会引发“栈溢出(stack smashing)”。
\section{线程共享进程数据}
\begin{itemize}
	\item 实验目的:了解线程与进程之间的数据共享关系。创建一个线程,在线程中更改进程中的数据。
	\item 实验软件环境: Ubuntu22.04 \& gnu-c++17
	\item 实验内容: 在进程中定义全局共享数据,在线程中直接引用该数据进行更改并输出该数据。
\end{itemize}
\subsection{代码实现}
在“\textbf{\textit{os2.cpp}}”中写如下代码。
\begin{lstlisting}[language=c++]
#include <thread>
#include <mutex>
#include <bits/stdc++.h>
#include <unistd.h>
using namespace std;
static int Sample = 1;
mutex mutx;
void thread1(int n){
	while (Sample <= n)
	{
		if (mutx.try_lock())
		{
			cout << "Thread1 " << Sample << "\n";
			sleep(1);
			Sample++;
			mutx.unlock();
		}
	}
}
int main(){
	int n = 20;
	thread t1(thread1, n);
	while (Sample <= n){
		if (mutx.try_lock()){
			cout << "MainThread " << Sample << "\n";
			Sample++;
			mutx.unlock();
		}
	}
	return 0;
}
	\end{lstlisting}
\subsection{实验结果}
运行上述代码,可得到实验结果如下图所示。
\begin{figure}
	\begin{center}
		\includegraphics[width=0.7\pdfpagewidth]{os2.png}
	\end{center}
\end{figure}
\subsection{实验结果分析}
在此次实验结果中,当\texttt{Sample}值为3时,执行了\texttt{thread1}函数。线程中直接引用了这一数据进行更改并输出之。
\subsection{反思}
本次实验代码中运行结果可能与“实验结果”章节中的不一致,有时会出现\texttt{thread1}函数不执行的情况,20个输出结果均为“MainThread=x”。
\end{document}