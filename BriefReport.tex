\documentclass[UTF8]{ctexart}
\usepackage{CJKutf8}
\usepackage{listings}
\usepackage{geometry}
\usepackage{xcolor}
\lstset{
	backgroundcolor=\color{white},
	basicstyle=\footnotesize\ttfamily,
	breakatwhitespace=false,
	breaklines=true,
	captionpos=b,
	commentstyle=\color{mygreen},
	deletekeywords={...},
	escapeinside={\%*}{*)},
	extendedchars=true,
	frame=single,
	keepspaces=true,
	keywordstyle=\color{blue},
	morekeywords={*,...},
	numbers=left,
	numbersep=5pt,
	numberstyle=\tiny\color{gray},
	rulecolor=\color{black},
	showspaces=false,
	showstringspaces=false,
	showtabs=false,
	stepnumber=4,
	stringstyle=\color{purple},
	tabsize=4,
	title=\lstname
}
\geometry{a4paper, left=1cm, right=1cm}
\title{操作系统实验报告}
\author{顾芃骐20079100001}
\begin{document}
	\maketitle
	\newpage
	\section{进程的建立}
	\begin{itemize}
		\item 实验目的:学会通过基本的Windows或者Linux进程控制函数,由父进程创建子进程,并实现父子进程协同工作。
		\item 实验软件环境:Ubuntu22.04 \& GNU-C++17
		\item 实验内容:创建两个进程,让子进程读取一个文件,父进程等待子进程读取。完文件后继续执行,实现进程协同工作。进程协同工作就是协调好两个进程,使之安排好先后次序并以此执行,可以用等待 函数来实现这一点。当需要等待子进程运行结束时,可在父进程中调用等待函数。
	\end{itemize}
	\subsection{代码实现}
\begin{lstlisting}[language=c++]
#include<fstream>
#include<unistd.h>
#include<stdio.h>
#include<iostream>
using namespace std;
int main(){
	int flag = 0;
	string answer = "";
	auto x = vfork();
	if (!x) {
		ifstream in("os1.in");
		in >> answer;
		in.close();
		flag = 1;
		cout << "End of Child Process" << endl;
	} else {
		while (!flag);
		cout << "Start of Main Process" << endl;
		cout << answer << endl;
		cout << "End of Main Process" << endl;
	}
	return EXIT_SUCCESS;
}
\end{lstlisting}
\subsection{实验结果}
运行上述代码,可得到如下实验结果:
\end{document}